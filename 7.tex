\chapter{Sistemi dinamici elementari}

\section{Parametri della risposta al gradino}
\begin{center}
\renewcommand{\arraystretch}{1.5}
  \begin{tabular}{|c|l|}
  \hline
    Simbolo & Descrizione \\
  \hline
    $S$ & Sovraelongazione \\
    $T_r$ & Tempo di ritardo \\
    $T_s$ & Tempo di salita \\
    $T_m$ & Tempo di massima sovraelongazione  \\
    $T_a$ & Tempo di assestamento  \\
  \hline

  \end{tabular}
\end{center}


\section{Sistemi del secondo ordine(senza zeri)}
Per risolvere questo esercizio conviene portare la funzione nella seguente forma:
\begin{align}
  G(s) \frac{\omega_n^2}{s^2+2\delta\omega_ns+\omega_n^2}, \quad G(0)=1
\end{align}

Ovviamente l'equazione differenziale che descrive il sistema è:
\begin{align}
  D^2y + 2\delta\omega_nDy+\omega_n^2y = \omega_n^2u
\end{align}

La pullsaizone naturale è $\omega_n$.

Per determinare la risposta al gradino unitario, si moltiplica la funzione di trasferimento per la trasformata di Laplace del gradino per la funzione di trasferimento,
Dopo di che di ottengono:

\begin{center}
\renewcommand{\arraystretch}{1.5}
  \begin{tabular}{|l|c|}
    \hline
      Pulsazione & $\omega = \omega_n\sqrt{1-\delta^2}$ \\
      Massima sovraelongazione(\%) & $S = 100 \exp\{- \frac{\delta \pi}{\sqrt{1 - \delta^2}}\}$ \\
      Tempo di assestamento & $T_a \approx \frac{3}{\delta \omega_n}$ \\
      Tempo di salita & $T_s \approx \frac{1.8}{\omega_n}$ \\
    \hline
  \end{tabular}
\end{center}


\section{Poli dominanti}
I poli dominanti sono quei poli(normalemnte una coppia) che non sono soggetti a quasi cancellazione 
polo-zero e sono più vicini all'asse immaginario.
