\chapter{Propriet\`a della trasformata $\mathcal{Z}$}



\section{Trasformata di un segnale ritardato di $n$ passi}
\begin{align}
  \mathcal{Z}[x(k-n)] = z^{-n} \mathcal{Z}[x(k)] + \sum_{k=0}^{n-1} x(k-n) z^{-k}
\end{align}



\section{Segnale anticipato di $n$ passi}

\begin{align}
  \mathcal{Z}[x(k+n)] = z^{n} \mathcal{Z}[x(k)] - \sum_{i=0}^{n-1} x(i)z^{n-i}
\end{align}



\section{Teoriema del valore iniziale}
\begin{enumerate}
  \item Se $X(z)$ non ha poli nell'origine $|z| \geq 1$, vuol dire che i poli sono 
  concentrati nel cerchio unitario
  \item Se $X(z)$ non ha poli nell'origine $|z| \geq 1$ eccetto un polo semplice in 
  $z=1$ allora $x(k)$ converge all'infinito
  \item Se $X(z)$ ha un polo multiplo in $z=1$ allora non converge 
  \item Se $X(z)$ ha un polo nella regione $|z| \geq 1, z \neq 1$ allora non converge
\end{enumerate}

\section{Trasformata $\mathcal{Z}$ di $a^k x(k)$}
\begin{align}
  \mathcal{Z}[a^k x(k)] = X(\frac{z}{a})
\end{align}

\section{Convoluzione a tempo discreto}
\begin{align}
  (x * y)(k) = x(k) * y(k) = \sum_{i=-\infty}^{+\infty} x(k-1)y(i)
\end{align}

La trasofrmata $\mathcal{Z}$ della convoluzione \`e:
\begin{align}
  \mathcal{Z}[x(k) * y(k)] = X(z)Y(z)
\end{align}

\section{Derivata della trasformata $\mathcal{Z}$}

\begin{align}
  -z \frac{d}{dz} X(z) = \mathcal{Z}[k x(k)]
\end{align}
