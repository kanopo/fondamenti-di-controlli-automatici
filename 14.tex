\chapter{Progetto di un sistema di controllo in retroazione}


\section{Specifiche di progetto}
\begin{enumerate}
  \item Buona connessione(numeratore minore del denominatore)
  \item Stabilit\`a asintotica (radici del denominatore a parte reale negativa)
  \item Prestazioni asintotiche
  \item Prestazioni dinamiche
  \item Robustezza
\end{enumerate}



\section{Regolatori standard}
i regolatori standard si basano sulla combinazione di:
\begin{itemize}
  \item Proporzionale
  \item Integrale
  \item Derivativo
\end{itemize}

La funzione di trasferimento di un controllore proporzionale-integrale-derivativo 
dove $R(s) = (1 + T_d s + \frac{1}{T_i s})$, sviluppando la funzione di trasferimento :
\begin{align}
  R(s) \approx K_p \frac{
    T_d T_i s^2 + T_i s + 1
  }{
    T_i s (1 + \tau s)
  }
\end{align}
  

\section{Rete ritardatrice}
Introduco:
\begin{align}
  C_r(s) = \frac{
    1 + \alpha \tau s
  }{
    1 + \tau s
  }
\end{align}


L'idea alla base della rete ritardatrice \`e quello di attenuare in alta frequenza
il guadagno ad anello, modificando il diagramma polare per ottenere la stabilit\`a.


La rete ritardatrice \`e un filtro passa basso, che attenua le alte frequenze.


Progettiamo $\alpha$ e $\tau$ in modo da assicurare la stabilit\`a asintotica
con un buon margine di ampiezza e fase.

\textbf{Con la rete ritardatrice il guadagno ad anello rimane elevato alle basse frequenze(buone prestazioni statiche)
ma riduce la banda passante(buone prestazioni dinamiche).}

\section{Rete anticipatrice}
\begin{align}
  C_a(s) = \frac{
    1 + \tau s
  }{
    1 + \alpha \tau s
  }
\end{align}



La rete anticipatrice ha diverse caratteristiche:
\begin{itemize}
  \item mantenimento delle prestazioni statiche
  \item stabilizzazione e allargamento banda passante del guadagno ad anello
  \item aumento del grado di stabilit\`a $G_s$
  \item diminuzione del tempo di assestamento $T_a$
  \item miglior capacit\`a di inseguiemento del segnale
\end{itemize}

Lo svantaggio \`e che la rete anticipatrice introduce rumore,
infatti $\alpha$ non pu\`o essere troppo piccolo.

% \section{Sintesi in frequenza}
%
% Avendo una rete anticipatrice:
% \begin{align}
%   C_a(s) = \frac{
%     1 + \tau s
%   }{
%     1 + \alpha \tau s
%   }
% \end{align}
% La converto in frequenza:
% \begin{align}
%   C_a(j \omega) = \frac{
%     1 + j \omega \tau
%   }{
%     1 + j \alpha \omega \tau
%   }
% \end{align}
%
% Posso calcolare modulo e fase:
%
% \begin{align}
%   M &= \frac{
%     \sqrt{1 + \omega^2 \tau^2}
%   }{
%     \sqrt{1 + \alpha^2 \omega^2 \tau^2}
%   }\\
%   \varphi &= \arctan(\omega \tau) - \arctan(\alpha \omega \tau)
% \end{align}
%
% Per effettuare l'inverso, avendo modulo e fase:
%
% \begin{align}
%   \alpha &= \frac{
%     M \cos \varphi - 1
%   }{
%     M (M - \cos \varphi)
%   } \\
%   \tau \omega &= \frac{
%     M - \cos \varphi
%   }{
%     \sin \varphi
%   }
% \end{align}
%
% \subsection{Rete anticipatrice e imposizione del margine di fase}
%
% Scegliere $\omega_0$ affinch\`e:
% \begin{align}
%   \varphi_0 = M_F - \arg L (j \omega_0) - \pi
% \end{align}
%
% valga:
% \begin{align}
%   \cos \varphi_0 > | L(j \omega_0) |
% \end{align}
%
% dopo aver definito:
% \begin{align}
%   M &= \frac{
%     1
%   }{
%     | L(j \omega_0) |
%   } \\
%   \varphi &= \varphi_0  \\
%   \tau &= \frac{
%     M - \cos \varphi
%   }{
%     \omega_0 \sin \varphi
%   } \\
%   \alpha &= \frac{
%     M \cos \varphi - 1
%   }{
%     M (M - \cos \varphi)
%   }
% \end{align}
%
% \begin{align}
%   M_F(\omega_0) = \pi + \arg L(j \omega_0) + \arccos | L(j\omega_0) |
% \end{align}
%
%
% \subsection{Rete ritardatrice e imposizione del margine di fase}
%
% Si sceglie $\omega_0$ affinch\`e:
% \begin{align}
%   \varphi_0 = \arg L(j \omega_0) + \pi - M_F
% \end{align}
%
% con $\varphi_0 > \frac{1}{|L(j\omega_0)|}$
%
% Segue che:
% \begin{align}
%   M = | L(j\omega_0) | \\
%   \varphi = \varphi_0 \\
%   \tau = \frac{
%     M - \cos \varphi
%   }{
%     \omega_0 \sin \varphi
%   } \\
%   \alpha = \frac{
%     M \cos \varphi - 1
%   }{
%     M (M - \cos \varphi)
%   }
% \end{align}
%
%
% \subsection{Rete anticipatrice e imposizione del margine di ampiezza}
%
% PAGINA 37 -> 41




