\section{Sistemi dinamici elementari} \label{sec:Sistemi dinamici elementari}

\subsection{Sistemi del primo ordine(strettamente propri)}


\begin{align}
  G(s) = \frac{1}{1 + \tau s}
\end{align}

Se applico alla funzione di trasferimento un ingresso a gradino ottengo la risposta diventa:
\begin{align}
  g_s(t) &= \mathcal{L}^{-1} \Big[ \frac{1}{1+\tau s} \cdot \frac{1}{s} \Big] \\
  &= 1 - e^{-\frac{t}{\tau}}
\end{align}

I parametri interessanti delle risposte al gradino sono:
\begin{itemize}
  \item $S$ Massima sovraelongazione (percentuale del valore di regime)
  \item $T_r$ Tempo di ritardo
  \item $T_s$ Tempo di salita
  \item $T_m$ Istante di massima sovraelongazione
  \item $T_a$ Tempo di assestamento
\end{itemize}


\subsection{Sistemi del secondo ordine(senza zeri)}

La funzione di trasferimento è:
\begin{align}
  G(s) = \frac{\omega_n^2}{s^2 + 2 \delta \omega_n s + \omega_n^2}, \quad G(0) = 1
\end{align}

Da notare che:
\begin{itemize}
  \item $\omega_n$ è la pulsazione naturale
  \item $\delta$ è il coefficiente di smorzamento
\end{itemize}

La risposta al gradino unitario è:
\begin{align}
  y(t) &= \mathcal{L}^{-1} \Big[  \frac{\omega_n^2}{s^2 + 2 \delta \omega_n s + \omega_n^2}  \cdot \frac{1}{s}\Big] \\
  &= 1 - \frac{1}{\sqrt{1-\delta^2}} e^{-\delta \omega_n t} \sin (\omega t + \varphi)
\end{align}


Si possono ricavare i seguenti parametri:
\begin{align}
  \omega &= \omega_n \sqrt{1-\delta^2} \\
  \varphi &= \arccos (\delta) = \arcsin (\sqrt{1 - \delta^2}) = \arctan \frac{\sqrt{1-\delta^2}}{\delta}
\end{align}



La massima sovrapposizione è:

\begin{align}
  S = 100 e^{-\frac{\pi \delta}{\sqrt{1-\delta^2}}}
\end{align}

Il tempo di assestamento è:
\begin{align}
  T_a \cong \frac{3}{\delta \omega_n}
\end{align}


Il tempo di salita è approssimato con:
\begin{align}
  T_s \approx \frac{1.8}{\omega_n}
\end{align}


\subsection{Poli dominanti di un sistema dinamico}
% Avendo un sistema generico con funzione di trasferimento:
% \begin{align}
%   G(s) = \frac{b(s)}{a(s)}
% \end{align}

% con $n$ poli e $m$ zeri, ecco un esempio per capire meglio:
% \begin{align}
%   G_1(s) = \frac{20}{(s+1)(s+4)(s+5)}
% \end{align}

% I \textbf{poli} della funzione di trasferimento sono:
% \begin{itemize}
%   \item $-1$
%   \item $-4$
%   \item $-5$
% \end{itemize}




\textbf{Sono i poli (normalmente una coppia), non soggetti a quasi cancellazione polo-zero, più vicini all'asse immaginario}

La risposa al gradino unitario dipende quasi esclusivamente dai soli poli dominanti del
sistema, se i poli sono complessi e coniugati posso usare le definizioni di $S$, $T_s$ e $T_a$ 
dei sistemi di ordine due.




