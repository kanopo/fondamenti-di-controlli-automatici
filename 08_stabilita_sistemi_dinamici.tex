\section{La stabilità dei sistemi dinamici}

\subsection{Definizioni e teoremi}

Considerando il sistema dinamico seguente:
\begin{align}
    \sum_{i=0}^{n} a_i D^i y = \sum_{i=0}^{m} b_i D^i u
\end{align}

La funzione di trasferimento è:
\begin{align}
    G(s) = \frac{b}{a}
\end{align}


Il guadagno statico è definito dal valore per $s=0$:
\begin{align}
    G(0) = \frac{b_0}{a_0}
\end{align}

% \begin{definition}[Perturbazione dell'evoluzione]
%     Data la coppia $(u(t), y(t)) \in \mathcal{B}$, una perturbazione di questa è caratterizzata
%     da una modifica nell'intervallo $[t_0, 0)$ sia dall'ingresso sia dall'uscita determinando la coppia 
%     perturbata $(\hat{u(t)}, \hat{y(t)}) \in \mathcal{B}$
% \end{definition}

\begin{definition}[stabilità (alle perturbazioni) di un sistema dinamico lineare]
    Il sistema si dice:
    \begin{itemize}
        \item \textbf{Stabile} se per ogni perturbazione la risposta libera è limitata su $[0, +\infty[$
        \item \textbf{Asintoticamente stabile} se è stabile e inoltre la risposta libera tende a zero per tempi infiniti
        \item \textbf{Semplicemente stabile} se è stabile e inoltre esiste una perturbazione per la quale la risposta libera non è limitata
        \item \textbf{Instabile} se non è stabile 
    \end{itemize}
\end{definition}


\begin{theorem}[Poli e stabilità]
    Sia un sistema lineare per il quale i poli coincidono con le radici del polinomio caratteristico
    ($a(s)$ e $b(s)$ sono coprimi fra loro), allora: 
    \begin{itemize}
        \item Il sistema è \textbf{stabile} se e solo se tutti i poli hanno parte reale non positiva e gli eventuali poli puramente immaginari sono semplici
        \item Il sistema è \textbf{asintoticamente stabile} se e solo se tutti i suoi poli hanno parte reale negativa
        \item Il sistema è \textbf{semplicemente stabile} se e sono se tutti i poli hanno parte reale non positiva e quelli puramente immaginari (che devono esistere) sono semplici
        \item Il sistema è \textbf{instabile} se e sono se esiste almeno un polo a parte reale positiva oppure un polo puramente immaginario con molteplicità maggiore di uno
    \end{itemize}
\end{theorem}


\subsection{Stabilità Bounded Input Bounded Output(BIBO)}

\begin{definition}[BIBO]
    Un sistema è BIBO stabile se per ogni azione forzante limitata la corrispondente
    risposta forzata è anch'essa limitata.
\end{definition}

\subsection{Criterio di Routh}
Considerando il sistema descritto dall'equazione differenziale:
\begin{align}
    \sum_{i=0}^{n}  a_i D^i y = \sum_{i=0}^{m}  b_i D^i u
\end{align}

\begin{definition}[Equazione caratteristica]
    Dato il sistema, l'equazione $a(s) = 0$ è detta equazione caratteristica del sistema.
\end{definition}

\begin{definition}[Polinomio di Hurwitz]
    Un polinomio $a(s)$ è detto di Hurwitz o hurwitziano se tutte le sue radici hanno parte reale negativa.
\end{definition}

È possibile determinare il segno delle radici di $a(s)$ mediante l'uso della \textbf{tabella di Routh}.

La tabella di Routh è una procedura algoritmica usata in controllo automatico per determinare la stabilità di un sistema dinamico lineare. Per un dato polinomio caratteristico, la tabella di Routh è costruita seguendo determinate regole per organizzare i coefficienti del polinomio. La stabilità del sistema è determinata dal segno dei coefficienti nella prima colonna della tabella: se tutti i coefficienti sono positivi, allora il sistema è stabile.

La base della tabella di Routh segue questo schema:
\begin{table}[h!]
    \centering
    \begin{tabular}{c | c c c c}
        $n$ & $x_{0, 0}$ & $x_{0, 1}$ & $x_{0, 2}$ & \dots \\
        $n-1$ & $x_{1, 0}$ & $x_{1, 1}$ & $x_{1, 2}$ & \dots \\
        $n-2$ & $x_{2, 0}$ & $x_{2, 1}$ & $x_{2, 2}$ & \dots \\
        \dotfill
    \end{tabular}
\end{table}

Le prime due righe della tabella di Routh sono i coefficienti di $a(s)$, ecco un esempio:
\begin{align}
    a_n s^n +
    a_{n-1} + s^{n-1} +
    a_{n-2} + s^{n-2} +
    a_{n-3} + s^{n-3} +
    a_{n-4} + s^{n-4} +
    a_{n-5} + s^{n-5} + 
    \dots
\end{align}

Vengono posizionati nelle prime due righe della tabella di Routh nel seguente modo:
\begin{table}[h!]
    \centering
    \begin{tabular}{c | c c c c}
        $n$ & $a_n$ & $a_{n-2}$ & $a_{n-4}$ & \dots \\
        $n-1$ & $a_{n-1}$ & $a_{n-3}$ & \dots & \dots
        \dotfill
    \end{tabular}
\end{table}
Mentre tutte le altre parti della tabella vengono calcolate come segue:
\begin{align}
    x_{i, j} &= - \frac{
        \begin{bmatrix}
            x_{i-2, 0} & x_{i-2, j+1} \\
            x_{i-1, 0} & x_{i-1, j+1}
        \end{bmatrix}
    }{x_{i-1, 0}} \\
    &= \frac{x_{i-1, 0} \cdot x_{i-2, j+1} - x_{i-2, 0} \cdot x_{i-1, j+1} }{x_{i-1, 0}}
\end{align}


\begin{definition}[Routh]
    Si assuma che la tabella di Routh possa essere completata.    
    Allora a ogni variazione di segno degli elementi della prima colonna,
    corrisponde a una radice con parte reale positiva.
\end{definition}

\begin{definition}[Criterio di Routh]
    Il polinomio $a(s)$ è hurwitziano se e solo se l'associata tabella di Routh
    può essere completata (con l'algoritmo di base) e presenta nella prima 
    colonna solo permanenze del segno.
\end{definition}

\subsection{Casi particolari nella costruzione della tabella di Routh}
Esistono due casistiche:
\begin{itemize}
    \item Primo elemento di una riga uguale a zero
    \item Tutti gli elementi di una riga uguali a zero
\end{itemize}


\subsubsection{Caso 1}
Si utilizza il metodo di \textbf{Benidir-Picinbono} che è sempre risolutivo.


\begin{theorem}[Metodo di Benidir-Picinbono]
    Ogni riga non nulla che inizia con $p$ zeri viene sommata con la riga da questa ottenuta
    moltiplicandola per $(-1)^p$ e traslandola verso sinistra di $p$ posizioni.
    La tabella di Routh viene poi continuata e interpretata nel modo usuale.
\end{theorem}


Segue un esempio:
\begin{align}
    a(s) = s^3 + 3s -2 = 0
\end{align}

\begin{table}[h!]
    \centering
    \begin{tabular}{c | c c c}
        $3$ & $1$ & $3$ & $0$ \\
        $2$ & $0$ & $-2$ & $0$ \\
    \end{tabular}
\end{table}

Visto che la riga numero due Inizia con $p=1$ zeri, allora devo sommare la riga due
con la riga due moltiplicata per $(-1)^1 = -1$ e traslata di uno posizione verso
sinistra:

\begin{table}[h!]
    \centering
    \begin{tabular}{c | c c c}
        $3$ & $1$ & $3$ & $0$ \\
        $2$ & $2$ & $-2$ & $0$ \\
    \end{tabular}
\end{table}

E poi si continua con il metodo di Routh tenendo in considerazione che 
il metodo procede come al solito.

\subsubsection{Caso 2}
Se tutti gli elementi di una riga sono uguali a zero, si deve definire un polinomio ausiliario
che prende la riva immediatamente precedente e si calcola la derivata di questo polinomio.

