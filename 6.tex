\chapter{Funzioni di trasferimento}

\section{Definizioni}
\subsection{Proprio}
Un sistema si dice \textbf{(strettamente) proprio} se la sua funzione di trasferimento
\`e \textbf{(strettamente) propria}.
Quindi con grado relativo $\rho \geq 0$ il sitema \`e proprio, mentre con grado relativo $\rho \geq 1$ il sistema \`e strettamente proprio.


\subsection{Guadagno statico}
Il guadagno statico \`e il valore:
\begin{align}
  K := \frac{y_c}{u_c}
\end{align}

Dove la $y_c$ \`e la risposta del sistema all'ingresso costante $u_c$.


\subsection{Polinomio caratteristico}

Dato il sistema $\sum$ descritto dall'equazione differenziale:
\begin{align}
  \sum_{i=0}^n a_i D^i y = \sum_{i=0}^m b_i D^i u
\end{align}



Il polinomio caratteristico \`e definito come:
\begin{align}
  a(s) = \sum_{i=0}^n a_i s^i
\end{align}


\subsection{Modi del sistema}

I modi sono le funzioni tipiche asociate ai poli del sistema,
se $p$ \`e un polo reale di molteplicit\`a $h$ i suoi modi saranno definito come:

\begin{align}
  e^{pt}, te^{pt}, t^2e^{pt}, \dots, t^{h-1}e^{pt}
\end{align}

Mentre se $p$ \`e un polo complesso coniugato($\sigma + j\omega$) di molteplicit\`a $h$ i suoi modi saranno definito come:

\begin{align}
  e^{\sigma t} \sin(\omega t + \phi_1), te^{\sigma t} \sin(\omega t + \phi_2), \dots, t^{h-1}e^{\sigma t} \sin(\omega t + \phi_h)
\end{align}
Che \`e equivalente a:
\begin{align}
  e^{\sigma t} \sin{\omega t}, e^{\sigma t} \cos{\omega t}, te^{\sigma t} \sin{\omega t}, te^{\sigma t} \cos{\omega t}, \dots, t^{h-1}e^{\sigma t} \sin{\omega t}, t^{h-1}e^{\sigma t} \cos{\omega t}
\end{align}

\subsection{Segnali tipici di ingresso}
\begin{center}
\renewcommand{\arraystretch}{1.5}
  \begin{tabular}{|l|c|c|}
    \hline
    Segnale & $u(t)$ & $U(s)$ \\
    \hline
    Impulso unitario & $\delta(t)$ & $1$ \\
    Gradino unitario & $1(t)$ & $\frac{1}{s}$ \\
    Rampa unitaria & $t(t)$ & $\frac{1}{s^2}$ \\
    Parabola unitaria & $\frac{1}{2}t^2(t)$ & $\frac{1}{s^3}$ \\
    \hline
  \end{tabular}
\end{center}

