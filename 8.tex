\chapter{Stabilit\`a dei sistemi dinamici}

\section{Stabilit\`a alle perturbazioni}
Bisogna analizzare i punti di equailibrio($G(0)$).


\begin{center}
\renewcommand{\arraystretch}{1.5}
  \begin{tabular}{|l|c|}
    \hline
      stabile & $y_{lib}(t)$ \`e limitata su $[0, +\infty)$ \\
      instabile & $y_{lib}(t)$ non \`e limitata su $[0, +\infty)$ \\
      asintoticamente stabile & $y_{lib}(t) \rightarrow 0$ per $t \rightarrow +\infty$ \\
      semplimente stabile & stabile ma esiste una perturbazione che lo rende instabile \\
      \hline
  \end{tabular}
\end{center}

\section{Poli e stabilitt\`a}


\begin{center}
\renewcommand{\arraystretch}{1.5}
  \begin{tabular}{|l|p{9cm}|}
    \hline
      stabile & $Re(p_i) \leq 0$ ed eventuali poli puramente immaginari semplici \\
      asintoticamente stabile & $Re(p_i) < 0$\\
      semplimente stabile & $Re(p_i) \leq 0$ e i poli puramente immaginari(devono essitere, al massimo uso $s=0$) sono semplici \\
      instabile & $Re(p_i) > 0$ oppure polo puramente immaginario con molteplicit\`a $> 1$  \\
    \hline
  \end{tabular}
\end{center}


\section{Stabilit\`a bounded-input bounded-output(BIBO)}

Un sistema \`e BIBO stabile se ogni ingresso limitato produce un'uscita limitata.

\section{Criterio di Routh-Hurwitz}

Considerando il solito sistema lineare $\sum$ descritto da $\sum_{i=0}^n a_iD^iy = \sum_{i=0}^m b_iD^iu $.

La funzione di trasferimento sar\`a $G(s) = \frac{b(s)}{a(s)}$.


Con il metodo di Routh si pu\`o analizzare la stabilitt\`a di un sistema
senza risolvere l'equazione caratteristica($a(s)=0$) e trovare i poli.


Il polinomio \`e hermitiano solo se i suoi coefficienti sono positivi, per fare questa dimostrazione
si riorre alla tabella di routh.

\subsection{Tabella di Routh}
la tabella di Routh \`e costituita da $n-1$ righe, calcolate a ritroso.

Una volta ordinato il polinomio in ordine decrescente di esponenti:
\begin{align}
  a_n s^n + a_{n-1} s^{n-1} + \dots + a_1 s + a_0 = 0
\end{align}


Si costruiscono le prime due righe della tabella alternando i coefficienti:
\begin{center}
  \begin{tabular}{|c|c|c|c|c|}
    \hline
    n & $a_n$ & $a_{n-2}$ & $a_{n-4}$ & $\dots$ \\
    \hline
    n-1 & $a_{n-1}$ & $a_{n-3}$ & $a_{n-5}$ & $\dots$ \\
    \hline
  \end{tabular}
\end{center}

Si possono calcolare tutte le righe succesive nella seguente maniera:

\begin{center}
  \begin{tabular}{|c|c|c|c|c|}
    \hline
    n & $\gamma_{0,0}$ & $\gamma_{0,1}$ & $\gamma_{0,2}$ & \dots \\
    n-1 & $\gamma_{1,0}$ & $\gamma_{1,1}$ & $\gamma_{1,2}$ & \dots \\
    n-2 & $\gamma_{2,0}$ & $\gamma_{2,1}$ & $\gamma_{2,2}$ & \dots \\
    \dotfill & \dotfill & \dotfill & \dotfill & \dotfill \\
    \hline
  \end{tabular}
\end{center}

Per calcolare $\gamma_{i,j}$ si usa la seguente formula:
\begin{align}
  \gamma_{i,j} = \frac{
    \begin{pmatrix}
      \gamma_{i-2, 0} & \gamma_{i-2, j+1} \\
      \gamma_{i-1, 0} & \gamma_{i-1, j+1}
    \end{pmatrix}
    }{\gamma_{i-1, 0}}
\end{align}

Super TIP: Immagina il calcola da fare come:
\begin{align}
  \frac{
    \begin{pmatrix}
      a & b \\
      c & d
    \end{pmatrix}
  }{c}
  =
  \frac{cb  - ad}{c}
\end{align}

Inoltre consiglio di tracciare una croce immaginaria sulla tabella di routh,
$c \rightarrow b \rightarrow a \rightarrow d \rightarrow c$.




\subsection{Teorema di Routh}
Si esamina la prima colonna della tabella calcolata e si osservgano le variazioni di segno nella prima colonna.

Si determinano il numero di variazioni e permanenze di segno(sommano a $n$).

Assumento di poter completare la tabella, \textbf{ad ogni variazione di segno corrisponde un polo con parte reale positiva}(causa instabilit\`a),
mentre ogni permanenza di segno corrisponde a un polo con parte reale negativa.


\subsection{Singolarit\`a della tabella}

Esistono due casi particolari che occorrono durante il calcolo della tabelle:
\begin{itemize}
  \item Il primo elemento di una riga \`e $0$
  \item Tutti gli elementi di una riga sono $0$
\end{itemize}

Per il primo caso si procede nel seguente modo:
ogni riga non nulla che inizia con $n$ zeri viene sommata con la riga ottenuta
moltiplicandola per $-1^n$ e traslandola verso sinistra di $n$ posizioni.


Se una riga \`e nulla, si procede nel seguente modo:

\begin{enumerate}
  \item Si sceglie come polinomio ausiliario quello della riga immediatamente sopra a quella con gli zeri
  \item Si deriva il polinomio ausiliario
  \item Sostituisco la riga di zeri con i coefficienti del polinomio ausiliario derivato
\end{enumerate}

Quando si far\`a il conteggio delle variazioni e delle permanenze, non ci saranno variazioni nella procedura.



