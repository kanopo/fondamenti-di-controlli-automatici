\section{Introduzione al controllo digitale}

\subsection{Convertitore A/D}

Campionamento del segnale con periodo $T$:
\begin{equation}
  x(t) \rightarrow \tilde{x(k)} = x(kT), \quad k \in \mathbb{Z}
\end{equation}

\subsection{Teorema del campionamento}

Un segnale $f(t)$ con spettro limitato da $0$ a $\omega_s$, si può ricostruire
dalla sequenza di campioni $\tilde{f}(k) = f(kT)$ se si campiona con frequenza
non inferiore a $2\omega_s$.

\subsection{Convertitore D/A}

Serve a convertire un segnale digitale in un segnale analogico, il più
cumune è il dispositivo di tenuta di ordine zero (ZOH).


\subsection{La trasformata zeta}

\begin{definition}[Trasformata zeta]
  Sia $x: \mathbb{Z} \rightarrow \mathbb{R}$ un segnale a tempo discreto.
  La trasformata zeta di $x(k)$ è :
  \begin{equation}
    \mathcal{Z}[x] \equiv \mathcal{Z}[x(k)] := \sum_{k=0}^{+\infty} x(k) z^{-k}
  \end{equation}
  Con $z$ variabile complessa.
\end{definition}



\begin{definition}[Linearità della trasformata zeta]
Dati due segnali $x(k)$ e $y(k)$ e due costanti $a$ e $b$:
\begin{equation}
  \mathcal{Z}[ax(k) + by(k)] = a \mathcal{Z}[x(k)] + b \mathcal{Z}[y(k)]
\end{equation}
\end{definition}


\begin{definition}[regione di convergenza o ROC]
  Data la sequenza $x(k)$, la regione di convergenza è definita come:
  \begin{equation}
  \text{ROC} := \{ z \in \mathbb{C} : | \sum_{k=0}^{\infty} x(k)z^{-k} | < \infty \}
  \end{equation}
\end{definition}


Le proprietà:
\begin{itemize}
  \item $X(z)$ è analitica su $\text{ROC}$
  \item Se $\text{ROC} \neq \emptyset$:
    \begin{equation}
      \text{ROC} = \{ z \in \mathbb{C} : |z| > \rho_c \}
    \end{equation}
    Dove $\rho_c$ è il raggio di convergenza.

  \item se $\rho > \rho_c$ la serie $\sum_{k=0}^{\infty} x(k)z^{-k}$ è convergente
  uniformemente su $\{ z \in \mathbb{C} : | z | \geq \rho \}$
\end{itemize}



