\chapter{Il controllo attivo di un processo}

\section{Definizioni}
\subsection{Behaviors}
Insieme di tutte le possibili coppie causa effetto assocaite ad un sistema.


\subsection{Linearit\`a}
Un'insieme si dice lineare quando soddisfa la propriet\`a di sovrapposizione degli effetti.


\subsection{Stazioneiet\`a}
Un sistema si dice stazionario quando il suo comportamento non cambia nel tempo.

\section{Controllo ad azione diretta e retroazione}

Il controllo attivo di un processo pu\`o essere realizzato in due modi:
\begin{itemize}
  \item azione diretta
  \item retroazione
\end{itemize}

Con il controllo ad azione diretta si ha che l'azione di comando dipende da:
\begin{itemize}
  \item obiettivo
  \item info sul processo
  \item ingressi
\end{itemize}


Con il controllo a retroazione si ha che l'azione di comando dipende da:
\begin{itemize}
  \item obiettivo
  \item info sul processo
  \item ingressi
  \item variabilli controllate
\end{itemize}



