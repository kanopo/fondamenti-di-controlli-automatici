\documentclass{article}

% Packages
\usepackage{amsmath} % For mathematical symbols and equations
\usepackage{amsthm} % For theorem environments
\usepackage{amsfonts} % For mathematical fonts
\usepackage{graphicx} % For including graphics

% Theorem environments
% \newtheorem{theorem}{Theorem}[section]
% \newtheorem{lemma}[theorem]{Lemma}
% \newtheorem{corollary}[theorem]{Corollary}
% \newtheorem{definition}[theorem]{Definition}

% Title and author
\title{Fondamenti di controlli automatici}
\author{Dmitri Ollari}

\begin{document}

\maketitle

\tableofcontents

\section{Analisi complessa}
Quando si ha a che fare con equazioni differenziali lineari, solitamente si preferisce trasformare il calcolo da dominio del tempo $t$ a dominio complesso $s$,
svolgere il calcolo come se fosse una problema polinomiale e ritornare nel dominio del tempo $t$ con il problema risolto.

\subsection{Poli}
Avendo una funzione d'esempio:
\begin{align}
  \frac{s(s+6)^3}{(s-2)(s+3)^2(s+5)^4}
\end{align}

I poli in questo caso sono:
\begin{itemize}
  \item $s = 2$ di ordine 1
  \item $s = -3$ di ordine 2
  \item $s = -5$ di ordine 4
\end{itemize}


\subsection{Zeri}
Avendo una funzione d'esempio:
\begin{align}
  \frac{s(s+6)^3}{(s-2)(s+3)^2(s+5)^4}
\end{align}

Gli zeri in questo caso sono:
\begin{itemize}
  \item $s = 0$ di ordine 1
  \item $s = -6$ di ordine 3
\end{itemize}

\section{Trasformata di Laplace}
La trasformata di laplace \`e comoda perch\`e permette di trasformare un problema differenziale in un problema algebrico e nel farlo mantiene la proprietà di linerià.
Il che permette di:
\begin{align}
  \mathcal{L}[c_1 f_1(t) + c_2 f_2(t)] = c_1 \mathcal{L}[f_1(t)] + c_2 \mathcal{L}[f_2(t)]
\end{align}

La trasformata di Laplace rispetta anche la proprietà di Iniettività, ovvero:
\begin{align}
  \mathcal{L}[f(t)] = \mathcal{L}[g(t)] \implies f(t) = g(t)
\end{align}

\renewcommand{\arraystretch}{1.5} % Aumenta la spaziatura verticale
\subsection{Trasformate di laplace interessanti}
\begin{center}
  
\begin{tabular}{| c | c |}
  \hline
  Segnale & Trasformata di Laplace \\
  \hline
  Gradino unitario & $\mathcal{L}[1(t)] = \frac{1}{s}$ \\
  \hline
  Segnale esponenziale & $\mathcal{L}[e^{at}] = \frac{1}{s-a}$ \\
  \hline
  & $\mathcal{L}[t^n] = \frac{n!}{s^{n+1}}$ \\
  \hline
  Derivata segnale impulsivo & $\mathcal{L}[\delta^{(n)}(t)] = s^n$ \\
  \hline
  Segnale impulsivo & $\mathcal{L}[\delta(t)] = 1$ \\
  \hline
\end{tabular}
\end{center}

\section{Funzione di trasferimento}
\subsection{Guadanagno statico}
\begin{align}
  Y(s) = G(0) U(s)
\end{align}

\subsection{polinomio caratteristico}
Dato:
\begin{align}
  \sum_{i=0}^n a_i D^i y = \sum_{i=0}^m b_i D^i u
\end{align}

Il polinomio caratteristico è:
\begin{align}
  a(s) = \sum_{i=0}^n a_i S^i
\end{align}


\subsection{Poli e Zeri}
I poli sono le radici di $a(s)$, gli zeri sono le radici di $b(s)$.


\subsection{Modi}
Se $p$ \`e un polo reale con molteplicit\`a $h$, allora:
\begin{align}
  t^{h-1} e^{pt}, t^{h-2} e^{pt}, \dots, e^{pt}
\end{align}

Fino ad arrivare ad avere $h = 0$.

Se $\sigma \pm j \omega$ \`e un polo complesso con molteplicit\`a $h$, allora:
\begin{align}
  t^{h-1} e^{\sigma t} \cos(\omega t), t^{h-1} e^{\sigma t} \sin(\omega t), \dots, e^{\sigma t} \cos(\omega t), e^{\sigma t} \sin(\omega t)
\end{align}

Oppure:
\begin{align}
  t^{h-1} e^{\sigma t} \sin(\omega t), t^{h-2} e^{\sigma t} \sin(\omega t), \dots, e^{\sigma t} \sin(\omega t)
\end{align}



Ecco un'esempio:
\begin{align}
  G(s) = \frac{(s+1)(S^2 + 2S + 7)}{(s+4)^4(s+5[(s+1^2 +4)])}
\end{align}


I modi sono:
\begin{align}
  \Big\{
    e^{-4t}, te^{-4t}, t^2e^{-4t}, t^3e^{-4t}, 
    e^{-5t},
    e^{-t} \sin(2t + \varphi_1) , t e^{-t} sin(2t + \varphi_2)
    \Big\}
\end{align}

\section{Sistemi dinamici elementari}

\begin{center}
  \begin{tabular}{| c | c |}
    \hline
    $S$ & Massima sovrapposizione \\
    $T_r$ & Tempo di ritardo \\
    $T_s$ & Tempo di salita \\
    $T_m$ & Istante di massima sovrapposizione \\
    $T_a$ & Tempo di assestamento \\
    \hline
  \end{tabular}
\end{center}


\end{document}
