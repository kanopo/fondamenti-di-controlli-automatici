\documentclass{article}

% ----------------------------------------------------
% PACKAGE
% ----------------------------------------------------

\usepackage{graphicx}
\usepackage{amsmath}
\usepackage{amsfonts}
\usepackage{amssymb}
\usepackage{hyperref}
\usepackage[a4paper, portrait, margin=0.75in]{geometry}
\usepackage[italian]{babel}
\usepackage{enumerate}% http://ctan.org/pkg/enumerate
\hypersetup{
    colorlinks=true,
    linkcolor=black,
    urlcolor=blue,
}

\title{Analisi dei dati \\[1ex] \large Un goliardico riassunto}
\author{Ollari Dmitri}

\begin{document}
    \maketitle
    

    \tableofcontents
    \listoffigures

    \newpage

\section{Analisi complessa}
Quando si ha a che fare con equazioni differenziali lineari, solitamente si preferisce trasformare il calcolo da dominio del tempo $t$ a dominio complesso $s$,
svolgere il calcolo come se fosse una problema polinomiale e ritornare nel dominio del tempo $t$ con il problema risolto.

\subsection{Poli}
Avendo una funzione d'esempio:
\begin{align}
  \frac{s(s+6)^3}{(s-2)(s+3)^2(s+5)^4}
\end{align}

I poli in questo caso sono:
\begin{itemize}
  \item $s = 2$ di ordine 1
  \item $s = -3$ di ordine 2
  \item $s = -5$ di ordine 4
\end{itemize}


\subsection{Zeri}
Avendo una funzione d'esempio:
\begin{align}
  \frac{s(s+6)^3}{(s-2)(s+3)^2(s+5)^4}
\end{align}

Gli zeri in questo caso sono:
\begin{itemize}
  \item $s = 0$ di ordine 1
  \item $s = -6$ di ordine 3
\end{itemize}

\section{Trasformata di Laplace}
La trasformata di laplace \`e comoda perch\`e permette di trasformare un problema differenziale in un problema algebrico e nel farlo mantiene la proprietà di linerià.
Il che permette di:
\begin{align}
  \mathcal{L}[c_1 f_1(t) + c_2 f_2(t)] = c_1 \mathcal{L}[f_1(t)] + c_2 \mathcal{L}[f_2(t)]
\end{align}

La trasformata di Laplace rispetta anche la proprietà di Iniettività, ovvero:
\begin{align}
  \mathcal{L}[f(t)] = \mathcal{L}[g(t)] \implies f(t) = g(t)
\end{align}

\renewcommand{\arraystretch}{1.5} % Aumenta la spaziatura verticale
\subsection{Trasformate di laplace interessanti}
\begin{center}
  
\begin{tabular}{| c | c |}
  \hline
  Segnale & Trasformata di Laplace \\
  \hline
  Gradino unitario & $\mathcal{L}[1(t)] = \frac{1}{s}$ \\
  \hline
  Segnale esponenziale & $\mathcal{L}[e^{at}] = \frac{1}{s-a}$ \\
  \hline
  & $\mathcal{L}[t^n] = \frac{n!}{s^{n+1}}$ \\
  \hline
  Derivata segnale impulsivo & $\mathcal{L}[\delta^{(n)}(t)] = s^n$ \\
  \hline
  Segnale impulsivo & $\mathcal{L}[\delta(t)] = 1$ \\
  \hline
  Segnale sinusoidale & $\mathcal{L}[\sin(a t)] = \frac{a}{s^2 + a^2}$ \\
  \hline
  Segnale cosinusoidale & $\mathcal{L}[\cos(a t)] = \frac{s}{s^2 + a^2}$ \\
  \hline
\end{tabular}
\end{center}

\section{Funzione di trasferimento}
\subsection{Guadanagno statico}
\begin{align}
  Y(s) = G(0) U(s)
\end{align}

\subsection{polinomio caratteristico}
Dato:
\begin{align}
  \sum_{i=0}^n a_i D^i y = \sum_{i=0}^m b_i D^i u
\end{align}

Il polinomio caratteristico è:
\begin{align}
  a(s) = \sum_{i=0}^n a_i S^i
\end{align}


\subsection{Poli e Zeri}
I poli sono le radici di $a(s)$, gli zeri sono le radici di $b(s)$.


\subsection{Modi}
Se $p$ \`e un polo reale con molteplicit\`a $h$, allora:
\begin{align}
  t^{h-1} e^{pt}, t^{h-2} e^{pt}, \dots, e^{pt}
\end{align}

Fino ad arrivare ad avere $h = 0$.

Se $\sigma \pm j \omega$ \`e un polo complesso con molteplicit\`a $h$, allora:
\begin{align}
  t^{h-1} e^{\sigma t} \cos(\omega t), t^{h-1} e^{\sigma t} \sin(\omega t), \dots, e^{\sigma t} \cos(\omega t), e^{\sigma t} \sin(\omega t)
\end{align}

Oppure:
\begin{align}
  t^{h-1} e^{\sigma t} \sin(\omega t), t^{h-2} e^{\sigma t} \sin(\omega t), \dots, e^{\sigma t} \sin(\omega t)
\end{align}



Ecco un'esempio:
\begin{align}
  G(s) = \frac{(s+1)(S^2 + 2S + 7)}{(s+4)^4(s+5[(s+1^2 +4)])}
\end{align}


I modi sono:
\begin{align}
  \Big\{
    e^{-4t}, te^{-4t}, t^2e^{-4t}, t^3e^{-4t}, 
    e^{-5t},
    e^{-t} \sin(2t + \varphi_1) , t e^{-t} sin(2t + \varphi_2)
    \Big\}
\end{align}

\section{Sistemi dinamici elementari}

\begin{center}
  \begin{tabular}{| c | c |}
    \hline
    $S$ & Massima sovrapposizione \\
    $T_r$ & Tempo di ritardo \\
    $T_s$ & Tempo di salita \\
    $T_m$ & Istante di massima sovrapposizione \\
    $T_a$ & Tempo di assestamento \\
    \hline
  \end{tabular}
\end{center}

\subsection{Sistemi del secondo ordine}

  \begin{align}
    G(s) = \frac{\omega_n^2}{s^2 + 2 \delta \omega_n s + \omega_n^2}
  \end{align}

  Dove $\omega_n$ \`e la pulsazione naturale e $\delta$ \`e il coefficiente di smorzamento.


  \section{Stabilit\`a dei sistemi dinamici}

  \begin{center}
    \begin{tabular}{| l | l |}
      \hline
      Stabile & Per tutte le perturbazioni, la risposta libera \`e limitata \\
      \hline
      Asintoticamente stabile & Per tutte le perturbazioni, la risposta libera tende a zero \\
      \hline
      Semplicemente stabile & Stabile ed esiste una perturbazione che fa divergere la risposta libera \\
      \hline 
      Instabile & Non \`e stabile \\
      \hline
    \end{tabular}
  \end{center}

  \subsection{Stabilit\`a poli e zeri}
  \begin{center}
    \begin{tabular}{| l | l |}
      \hline
        Stabile & Tutti i poli hanno parte reale non positiva, eventuali poli \\
        & puramente immaginari sono semplici \\

      \hline
        Asintoticamente stabile & Tutti i poli hanno parte reale negativa \\
      \hline
        Semplicemente stabile & Tutti i poli hanno parte reale non positiva e \\ 
        & quelli puramente immaginari (che devono esistere) sono semplici \\

      \hline
        Instabile & Esiste almeno un polo con parte reale positiva \\
        & oppure esiste un polo puramente immaginario con molteplicit\`a maggiore di uno \\

      \hline
    \end{tabular}
  \end{center}

  \subsection{Teorema di Routh}
  Assunto che riesco a completare la tabella di Routh, ad ogni variazione di segno dell'elemento nella colonna 1, corrisponde
  una radice con parte reale positiva,
  
  Vale lo stesso principio anche per il numero di radici reali negative, ad ogni continuazione dedl segno della elemento nella colonna 1,
  corrisponde una radice negativa.



  \subsubsection{Tabella}
  Per la costruzione della tabelli mi aiuto con un polinomio caratteristico del tipo:
  \begin{align}
    (s-1)(s^2+4s+5) + k(s+1)  = \\
    s^3 + 3s^2 + (k + 1)s + k - 5
  \end{align}

  Per prima cosa si inserisce nella tabella il polinomio di partenza.

  \begin{center}
    \begin{tabular}{| c | c  c  c |}
      \hline
      3 & 1 & k + 1 & 0 \\
      2 & 3 & k - 5 & 0 \\
      1 & $2k + 8$ & 0 & 0 \\
      0 & $k - 5$ & 0 & 0 \\
      \hline
    \end{tabular}
  \end{center}

  Una volta inserite nella tabella le prime due righe, si procede con il calcolo delle righe successive.
  Si utilizzeranno la seguente tabella di riferimento:

  \begin{center}
    \begin{tabular}{| c | c  c  c |}
      \hline
      3 & $\gamma_{0, 0}$ & $\gamma_{0, 1}$ & $\gamma_{0, 2}$ \\
      2 & $\gamma_{1, 0}$ & $\gamma_{1, 1}$ & 0 \\
      1 & $\gamma_{2, 0}$ & 0 & 0 \\
      0 & $\gamma_{3, 0}$ & 0 & 0 \\
      \hline
    \end{tabular}
  \end{center}

  Dove $\gamma_{i, j}$ \`e definito come:
  \begin{align}
    \gamma_{i, j} = \frac{
      \begin{pmatrix}
        \gamma_{i-2, 0} & \gamma_{i-2, j+1} \\
        \gamma_{i-1, 0} & \gamma_{i-1, j+1}
      \end{pmatrix}
    }{\gamma_{i-1, 0}}
  \end{align}


  Un metodo visuale per ricordare come moltiplicare e dividere i valori della tabella \`e il seguente:
  \begin{align}
    \frac{
      \begin{pmatrix}
        a & b \\
        c & d \\
      \end{pmatrix}
    }{c}
  \end{align}


  Quindi per poter dire che il sistema risulta stabile dobbiamo confermare che gli elementi della 
  prima colonna mantenga il segno.

  \begin{align}
    &\begin{cases}
      2k + 8 > 0 \\
      k-5 > 0
    \end{cases}
    \\
    &\begin{cases}
      k > -4 \\
      k > 5
    \end{cases}
  \end{align}
  Quindi il sistema retrazionato \`e stabile se e solo se $k > 5$.




  \section{Analisi armonica e Bode}

\end{document}
