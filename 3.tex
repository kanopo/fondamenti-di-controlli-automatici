\chapter{Cenni di analisi complessa}


\section{Poli}

\begin{align}
  f(s) = \frac{s(s+6)^3}{(s-2)(s+3)^2(s+5)^4}
\end{align}

I poli in questo esempio sono:
\begin{itemize}
  \item $2$ \`e un polo di ordine $1$
  \item $-3$ \`e un polo di ordine $2$
  \item $-5$ \`e un polo di ordine $4$
\end{itemize}

\section{Zeri}

\begin{align}
  f(s) = \frac{s(s+6)^3}{(s-2)(s+3)^2(s+5)^4}
\end{align}

Gli zeri in questo esempio sono:
\begin{itemize}
  \item $0$ \`e uno zero di ordine $1$
  \item $-6$ \`e uno zero di ordine $3$
\end{itemize}
