\chapter{Diagrammi di Nyquist e sistemi a fase minima}


I diagrammi di Nyquist sono utili per lo studio della stabilit\`a.

I comportamenti pi\`u importanti sono:
\begin{itemize}
  \item $\omega \rightarrow +\infty$
  \item $\omega \rightarrow 0+$ di tipo 0(no poli nell'origine)
  \item $\omega \rightarrow 0+$ di tipo  magiore uguale 1(polo nell'origine e dipende dalla molteplicit\`a)
\end{itemize}


\section{Sistemi a fase minima}
Considerp il sistema lineare e stazionario $\sum$ con funzione di trasferimento $G$ 
e risposta armonica $G(j\omega)$.


$\sum$ \`e a fase minima se il diagramma delle fase $\beta = \arg G(j\omega)$ \`e determinato 
univocamente(modulo $2\pi$) dal diagramma dei moduli $\alpha = |G(j\omega)|$ mediante la formula di Bode.







\section{Formula di}
\begin{align}
  \alpha = \ln |G(j\omega)| \\
  \beta = \arg G(j\omega)
\end{align}




