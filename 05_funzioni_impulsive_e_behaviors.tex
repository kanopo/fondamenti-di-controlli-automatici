\section{Le funzioni impulsive e l'insieme dei behaviors}

\subsection{Le funzioni impulsive}

Gradino unitario:
\begin{align}
	1(t) := \begin{cases}
		        0 & t < 0    \\
		        1 & t \geq 0
	        \end{cases}
\end{align}

Introduco la funzione $f(t;\tau) \in \mathcal{C}^0$:
\begin{align}
	f(t;\tau) := \begin{cases}
		             0               & t < \tau        \\
		             \frac{1}{\tau}t & 0 \leq t < \tau \\
		             1               & t > \tau
	             \end{cases}
\end{align}

Formalmente si definisce la \textbf{delta di Dirac} come:
\begin{align}
	\delta(t) := \lim_{\tau \to 0} Df(t;\tau)
\end{align}

Ricordo che la derivata di una funzione $f(t:\tau)$ è:

\begin{align}
	Df(t;\tau) := \begin{cases}
		              0              & t < \tau        \\
		              \frac{1}{\tau} & 0 \leq t < \tau \\
		              1              & t > \tau
	              \end{cases}
\end{align}


\subsection{Derivata generalizzata di una funzione discontinua}

$f \in \mathcal{C}^{+\infty}_p(\mathbb{R})$ e sia $t = 0$ il solo istante di discontionuità di $f$.

Definisco la derivata generalizzata di $f$ come:
\begin{align}
	g(t) & := \begin{cases}
		          f(t)               & t < 0    \\
		          f(t) - (f_+ - f_-) & t \geq 0 \\
	          \end{cases} \\
	g(t) & \in \mathcal{C}^0
\end{align}


Ovvero che la \textbf{funzione discontinua} $=$ \textbf{funzione continua} $+$ \textbf{funzione a gradino}.

Usando la derivata generalizzata:
\textbf{Derivata generalizzata di ordine 1} $=$ \textbf{funzione discontinua} $+$ \textbf{funzione impulsiva(ordine 0)}.



Principio di identità delle funzioni impulsive:
Le funzione impulsiva:
\begin{align}
  c_{-1} + c_0\delta(0) + c_1\delta'(0) + c_2\delta''(0) + \dots + c_n\delta^{(n)}(0)
\end{align}

E la funzione impulsiva:
\begin{align}
  d_{-1} + d_0\delta(0) + d_1\delta'(0) + d_2\delta''(0) + \dots + d_m\delta^{(m)}(0)
\end{align}

Sono uguale se e solo se:
\begin{align}
  c_i = d_i \quad \forall i = 0, 1, \dots, \min \{ n, m \}
\end{align}

Se $n > m$ allora $c_i = 0, i = m + 1, \dots, n$, altrimenti \\
se $m > n$ allora $d_i = 0, i = n + 1, \dots, m$.
