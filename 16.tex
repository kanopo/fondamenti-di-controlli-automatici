% \chapter{Regolatori PID}
%
% Ricordo che la funzione di trasferimento di un regolatore PID è:
%
% \begin{align}
% R(s) = K_p \Bigg( 1 + T_d s + \frac{1}{T_i s} \Bigg)
% \end{align}
%
% Riscrivo $R(s)$ per ottenere $\omega_n$ e $\delta$:
%
% \begin{align}
%   R(s) &= K_p \Bigg( 1 + \frac{T_d}{T_i} \frac{1}{s} + \frac{1}{T_i s} \Bigg) \\
%   &= \frac{K_p}{T_i} \frac{
%     T_d T_i s^2 + T_i s + 1
%   }{
%     s
%   } \\
%   &= \frac{K_p}{T_i} \frac{
%     1 + 2 \delta \frac{s}{\omega_n} + \frac{s^2}{\omega_n^2}
%   }{
%     s
%   } \\
% \end{align}
%
% Quindi $\omega_n$ e $\delta$ sono:
% \begin{align}
%   \omega_n &= \frac{1}{\sqrt{T_d T_i}} \\
%   \delta &= \frac{1}{2} \sqrt{\frac{T_i}{T_d}}
% \end{align}
%
% Passando alle frequenze, si ha che:
% \begin{align}
%   R(j\omega) = \frac{K_p}{T_i} \frac{2\delta}{\omega_n} = K_p
% \end{align}
%
% Mentre se la frequenza \`e $\omega_n$:
%
% \begin{align}
%   R(j\omega_n) = \frac{K_p}{T_i} \frac{1}{\omega_n^2} = \frac{K_p}{T_d} = K_p
% \end{align}
%
%
% \section{Implementazione regolatori PID}
%
