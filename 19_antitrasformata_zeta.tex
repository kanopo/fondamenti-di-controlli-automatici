\section{Antitrasformata zeta}

\subsection{Teorema dei residui}

Sia $\Gamma$ una curva chiusa semplice e $f(z)$ una funzzione analitica su
$\Gamma$ tranne che in un numero finito di punti $p_1, p_2, \dots, p_n$ allora:
\begin{equation}
  \oint_{\Gamma} f(z) dz = 2 \pi j \sum_{1=1}^{n} Res(f, p_1)
\end{equation}

Sia $l \in \mathbb{Z}$ allora:
\begin{equation}
  \oint_{\Gamma} z^l dz = \begin{cases}
    2 \pi j & l = -1 \\
    0 & l \neq -1
    \end{cases}
\end{equation}

\begin{definition}[Antitrasformata zeta]
  Sia $X(z)$ la trasformata zeta di una sequenza $x(k)$ allora:
  \begin{equation}
    x(k) = \frac{1}{2\pi j} \oint_{\gamma} X(z) z^{k-1} dz
  \end{equation}
  dove $\gamma$ è una curva chiusa semplice, percorsa in senso antiorario, che
  circonda la regione di non convergenza di $\mathcal{Z}[x(k)]$.

  Formalmente:
  \begin{equation}
    x(k) = \mathcal{Z}^{-1}[X(z)] = \frac{1}{2\pi j} \oint_{\gamma} X(z) z^{k-1} dz
  \end{equation}
\end{definition}


\subsection{Antitrasformata zeta con il metodo dei residui}

\begin{equation}
  x(k) = \frac{1}{2 \pi j} \cdot 2 \pi j \sum_{i} \text{Res}\{ X(z) \cdot z^{k-1}, p_i \}
\end{equation}

Si semplifica il tutto con la formula:
\begin{equation}
  x(k) = \sum_{i} \text{Res}\{ X(z) \cdot z^{k-1}, p_i \}
\end{equation}


Sfruttando una proprietà dei residui:
\begin{equation}
  \sum_{i=1}^{k} \text{Res}\{ F, p_i \} = \begin{cases}
    0 & \text{se } n-m > 1 \\
    \frac{b_m}{a_n} & \text{se } n-m = 1
  \end{cases}
\end{equation}


\subsubsection{Formule}
\begin{equation}
  \mathcal{Z}^{-1} \left[ \frac{1}{(z - a)^n} \right] = \frac{(k-1)(k-2) \dots (k - n + 1)}{ (n-1)! } a^{k-n} 1(k - 1)
\end{equation}

\begin{equation}
  \mathcal{Z}^{-1} \left[ \frac{z}{(z - a)^n} \right] = \frac{k(k-1)(k-2) \dots (k - n + 2)}{ (n-1)! } a^{k+1-n} 1(k)
\end{equation}







\subsubsection{Alcune trasformate calcolate con i residui}
\begin{equation}
  \mathcal{Z}^{-1} [ \frac{1}{z-a} ] = a^{k-1} 1(k-1)
\end{equation}

\begin{equation}
  \mathcal{Z}^{-1} \left[ \frac{1}{(z-a)^n} \right] = \frac{1}{(n-1)!} D^{n-1} \left[ z^{k-1} \right]_{z=a}
\end{equation}

Alcuni casi particolari sono:
\begin{equation}
  \mathcal{Z}^{-1} \left[ \frac{1}{z-a} \right] = a^{k-1} 1(k-1)
\end{equation}

\begin{equation}
  \mathcal{Z}^{-1} \left[ \frac{1}{(z-a)^2} \right] = (k-1) a^{k-2} 1(k-1)
\end{equation}

\begin{equation}
  \mathcal{Z}^{-1} \left[ \frac{1}{(z-a)^3} \right] = \frac{(k-1)(k-2)}{2} a^{k-3} 1(k-1)
\end{equation}



\begin{equation}
  \mathcal{Z}^{-1} \left[ \frac{z}{z-a} \right] = a^{k} 1(k)
\end{equation}

\begin{equation}
  \mathcal{Z}^{-1} \left[ \frac{z}{(z-a)^2} \right] = (k) a^{k-1} 1(k)
\end{equation}

\begin{equation}
  \mathcal{Z}^{-1} \left[ \frac{z}{(z-a)^3} \right] = \frac{(k)(k-1)}{2} a^{k-2} 1(k)
\end{equation}



\subsection{Antitrasformata zeta con il metodo dello sviluppo in fratti semplici}

\begin{equation}
  F(z) = \frac{
    b(z)
  }{
    (z - p_1)^{r_1} (z - p_2)^{r_2} \dots (z - p_h)^{r_h}
  }
\end{equation}

$p_i \neq p_j$ se $i \neq j$.


\begin{equation}
  \sum_{i=1}^{h} r_i = n, \quad \deg(b(z)) < n)
\end{equation}

\begin{equation}
  F(z) = c_0 + \frac{c_{1,1}}{(z-p1)^{r_1}} + \dots + \frac{c_{1,r_1}}{(z-p_1)} + \dots + \frac{c_{h,1}}{(z-p_h)^{r_h}} + \dots + \frac{c_{h,r_h}}{(z-p_h)}
\end{equation}

\begin{equation}
  c_0 = \lim_{z \to \infty} F(z)
\end{equation}


\begin{equation}
  c_{i, j} = \frac{1}{
    (j-i)!
  } D^{j-1} \left[ (z-p_i)^{r_i} F(z) \right] |_{z=p_i} \quad i = 1, \dots, h \quad j = 1, \dots, r_i
\end{equation}

