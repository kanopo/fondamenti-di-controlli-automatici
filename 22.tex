\chapter{Antitrasformata $\mathcal{Z}$}


\section{Antitrasformata metodo residui}
\begin{align}
  \mathcal{Z}^{-1}\Bigg[\frac{1}{z-a}\Bigg] &= a^{k-1} \cdot 1(k-1) \\
  \mathcal{Z}^{-1}\Bigg[ \frac{1}{(z-a)^2} \Bigg] &= (k-1) a^{k-2} \cdot 1(k-1) \\
  \mathcal{Z}^{-1}\Bigg[ \frac{1}{(z-a)^3} \Bigg] &= \frac{(k-1)(k-2)}{2} a^{k-3} \cdot 1(k-1) \\
\end{align}

\begin{align}
  \mathcal{Z}^{-1}\Bigg[\frac{z}{z-a}\Bigg] &= a^k \cdot 1(k) \\
  \mathcal{Z}^{-1}\Bigg[\frac{z}{(z-a)^2}\Bigg] &= k a^{k-1} \cdot 1(k) \\
  \mathcal{Z}^{-1}\Bigg[\frac{z}{(z-a)^3}\Bigg] &= \frac{k(k-1)}{2} a^{k-2} \cdot 1(k) \\
\end{align}


\section{Antitrasformata fratti complessi}

\begin{align}
  \mathcal{Z}^{-1}\Bigg[ \frac{c}{z-p} + \frac{\bar{c}}{z-\bar{p}} \Bigg] &= 2 | c | | p |^{k-1} \cos [ \arg(p) (k-1) + \arg(c) ] \cdot 1(k-1) \\
  \mathcal{Z}^{-1}\Bigg[ c\frac{c}{z-p} + \bar{c}\frac{\bar{c}}{z-\bar{p}} \Bigg] &= 2 | c | | p |^{k} \cos [ \arg(p) (k) + \arg(c) ] \cdot 1(k) \\
\end{align}


\section{Antitrasformata metodo fratti semplici}
\begin{align}
  \mathcal{Z}\Bigg[\binom{k}{n-1} a^{k-(n-1)}\Bigg] = \frac{z}{(z-a)^n}
\end{align}

