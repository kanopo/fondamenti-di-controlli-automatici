\section{Cenni di analisi complessa}

\subsection{Poli}
Avendo la funzione:
\begin{align}
	f(s) = \frac{s(s+6)^3}{(s-2)(s+3)^2(s+5)^4}
  \label{eq:funzione_esempio_poli_e_zeri}
\end{align}

I poli sono quei valori di $s$ per i quali il denominatore si annulla, quindi:
\begin{itemize}
  \item $s = 2$ è un polo semplice
  \item $s = -3$ è un polo doppio
  \item $s = -5$ è un polo quadruplo
\end{itemize}

\subsection{Zeri}
Usando la funzione della \autoref{eq:funzione_esempio_poli_e_zeri}, gli zeri sono quei valori di $s$ per i quali il numeratore si annulla, quindi:
\begin{itemize}
  \item $s = 0$ è uno zero semplice
  \item $s = -6$ è uno zero triplo
\end{itemize}




