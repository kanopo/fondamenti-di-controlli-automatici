\section{Sistemi a tempo discreto}

\begin{definition}[Sistema a tempo discreto]
  Un sistema a tempo discreto è un processo per il quale assegnato un segnale
  di ingresso a tempo discreto corrisponde un segnaledi uscita a tempo discreto.
\end{definition}

\begin{definition}[Linearità]
  Un sistema si dice linere quando soddisfa la proprietà di sovrapposizione
  degli effetti.  
\end{definition}


\begin{definition}[Stazionarietà]
  Un sistema di dice stazionrio (invariante nel tempo) se per ogni ingresso
  con una certa caratteristica, l'uscita ha la stessa caratteristica.

  \begin{equation}
    (u(k), y(k)) \in \mathcal{B}_d \implies (u(k + k_0), y(k + k_0)) \in \mathcal{B}_d
  \end{equation}
\end{definition}


\subsection{Risposta di un sistema lineare a tempo discreto e la funzione di trasferimento}


Definita l'equazione alle differenze come:
\begin{equation}
  a_n y(k) + a_{n-1} y(k-1) + \dots + a_0 y(k-n) = b_m u(k - n + m)  + \dots + b_0 u(k-n)
\end{equation}

\begin{itemize}
  \item $a_n \neq 0$
  \item $b_m \neq 0$
  \item $\{ a_0 \neq 0 \} \vee \{ b_0 \neq 0\}$
  \item $m \leq n$
\end{itemize}
