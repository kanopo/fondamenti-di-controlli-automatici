\section{La funzione di trasferimento}
\subsection{Estensione dell'insieme dei behaviors}

Dato un sistema dinamico $\sum$ descritto dall'equazione differenziale:
\begin{align}
  \sum_{i=0}^n a_i D^i y = \sum_{i=0}^m b_i D^i u
\end{align}

Si definisce estensione impulsiva dei behaviors o \textbf{behavior esteso} di $\sum$:
\begin{align}
  \mathcal{B}^* := \{ (u, y) \in \mathcal{C}_p^{\infty} (\mathbb{R})^* \times \mathcal{C}_p^{\infty} (\mathbb{R})^* : \sum_{i=0}^n a_i D^{*i} y = \sum_{i=0}^m b_i D^{*i} u \}
\end{align}

\begin{align}
  \mathcal{B} \subset \mathcal{B}^*
\end{align}

Se ne ricava una proprietà importante:
\begin{align}
  (u, y) \in \mathcal{B}^* \Rightarrow (D^* u, D^*y) \in \mathcal{B^*}
\end{align}


\subsection{Problema fondamentale dell'analisi nel dominio del tempo di un sistema $\sum$}

Avendo l'equazione differenziale che descrive il sistema $\sum$:
\begin{align}
  \sum_{i=0}^n a_i D^{*i} y(t) = \sum_{i=0}^m b_i D^{*i} u(t)
\end{align}

\textbf{Si deve applicare la trasformata di Laplace}:
\begin{align}
  \sum_{i=0}^n a_i \mathcal{L}[  D^{*i} y(t) ] &= \sum_{i=0}^m b_i \mathcal{L}[ D^{*i} u(t) ] \\
  \sum_{i=0}^n a_i s^i Y(s) &= \sum_{i=0}^m b_i s^i U(s)
\end{align}

Si definisce funzione di trasferimento del sistema $\sum$:
\begin{align}
  G(s) := \frac{Y(s)}{U(s)} = \frac{\sum_{i=0}^m b_i s^i}{\sum_{i=0}^n a_i s^i}
  \label{eq:funzione_trasferimento}
\end{align}



\subsubsection{Sistema proprio} % (fold)
\label{sec:Sistema proprio}
Un sistema si dice \textbf{(strettamente)proprio} se la sua funzione di trasfermento è una funzione razionale (strettamente)propria, cioè:
\begin{itemize}
  \item $n \geq m \implies \rho \geq 0$ il sistema è proprio
  \item $n > m \implies \rho \geq 1$ il sistema è strettamente proprio
\end{itemize}
% subsubsection Sistema proprio (end)


\subsubsection{Guadagno statico} % (fold)
\label{sec:Guadagno statico}

Rapporto fra valore costante dell'uscita e il valore costante dell'ingresso:
\begin{align}
  K := \frac{y_c}{u_c}
\end{align}

% subsubsection Guadagno statico (end)

\subsubsection{Polinomio caratteristico} % (fold)
\label{sec:Polinomio caratteristico}

Dato il sistema $\sum$:
\begin{align}
  \sum_{i=0}^n a_i D^{i} y = \sum_{i=0}^m b_i D^{i} u
\end{align}

Il polinomio caratteristico è:
\begin{align}
  a(s) := \sum_{i=0}^n a_i s^i
\end{align}

% subsubsection Polinomio caratteristico (end)

\subsubsection{Poli e Zeri} % (fold)
\label{sec:Poli e Zeri}
Avendo una funzione di trasferimento $G(s)$:
\begin{align}
  G(s) = \frac{b(s)}{a(s)}
\end{align}

gli zeri sono i valori di $s$ che annullano il numeratore $b(s)$, i poli sono i valori di $s$ che annullano il denominatore $a(s)$.
% subsubsection Poli e Zeri (end)

\subsubsection{Modi} % (fold)
\label{sec:Modi}
Se $p$ è un polo \textbf{reale} con molteplicità $h$:
\begin{align}
  e^{pt}, te^{pt}, \dots, t^{h-1}e^{pt}
\end{align}

Se $\sigma \pm j \omega$ è un polo \textbf{complesso} coniugato con molteplicità $h$:
\begin{align}
  e^{\sigma t} \sin(\omega t + \phi_1), te^{\sigma t} \sin(\omega t + \phi_2), \dots, t^{h-1}e^{\sigma t} \sin(\omega t + \phi_h)
\end{align}

oppure:
\begin{align}
  e^{\sigma t} \sin(\omega t), e^{\sigma t} \cos(\omega t), \dots, t^{h-1}e^{\sigma t} \sin(\omega t), t^{h-1}e^{\sigma t} \cos(\omega t)
\end{align}
% subsubsection Modi (end)


\subsubsection{Risposte canoniche} % (fold)
\label{sec:Risposte canoniche}
Usualmente le risposte canocniche prese in considerazione sono:
\begin{itemize}
  \item $g(t) \equiv$ risposta all'impulso $\delta(t)$
  \item $g_s(t) \equiv$ risposta al gradino $1(t)$
\end{itemize}
% subsubsection Risposte canoniche (end)
