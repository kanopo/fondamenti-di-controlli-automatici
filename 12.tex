\chapter{Sistemi retroazionati}
Avendo un sistema retrazionato definito da:
\begin{align}
  T(s) = \frac{G(s)}{1+G(s)H(s)}
\end{align}

Il sistema retroazionato \`e ben connesso se:
\begin{align}
  \lim_{|s| \to \infty} G(s)H(s) \neq 0
\end{align}

e la buona connessione \`e necessaria per la stabilit\`a del sistema.



\section{Criterio di Nyquist}
Metodo utile per lo studio della stabilit\`a asintotica.

Il criterio afferma che il sistema retroazionato \`e stabile asintoticamente
se e solo se l'equazione caratteristica $1 + G(s)H(s) = 0$ ha tutte le radici a parte reale 
negativa.

\textbf{Dopo di che il criterio di Nyquist richiede il tracciamento del diagramma di Nyquist di $G(j\omega)H(j\omega)$}.





\subsection{Teorema dell'indice logaritmico}

Se il diagramma di Nyquist ha senso \textbf{orario}:
\begin{align}
  \{\text{numero di giri \textbf{orari} della curva immagine intorno all'origine}\} = n_z - n_p
\end{align}


Mentre se il diagramma di Nyquist ha senso \textbf{antiorario}:

\begin{align}
  \{\text{numero di giri \textbf{antiorari} della curva immagine intorno all'origine}\} = n_z - n_p
\end{align}


Applicando il terorema alla stabilit\`a dei sistemi retroazionati:
\begin{itemize}
  \item $n_z$ \`e il numero di zeri di $1 + G(s)H(s)$ appartenenti a $\mathbb{C}$
  \item $n_p$ \`e il numero di poli di $1 + G(s)H(s)$ appartenenti a $\mathbb{C}$
  \item $\psi$ \`e il numero di giri in senso orario
\end{itemize}
